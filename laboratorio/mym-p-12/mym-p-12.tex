%%%%%%%%%%%%%%%%%%%%%%%%%%%%%%%%%%%%%%%%%%%%%%%%%%%%%%%%%%%%%%%%%%%%%%%%%%%%%%%%%%%%%%%%%
% Autor:        Aguilar Enriquez, Paul Sebastian a.k.a. Penserbjorne
% Fecha:        05/02/2017
% Descripcion:  Plantilla base para actividades o tareas.
%%%%%%%%%%%%%%%%%%%%%%%%%%%%%%%%%%%%%%%%%%%%%%%%%%%%%%%%%%%%%%%%%%%%%%%%%%%%%%%%%%%%%%%%%

\documentclass[a4paper,11pt]{article}                 % Papel tamaño carta, texto de 11pt.

\usepackage[top=2cm, bottom=2cm, left=2.2cm, right=2.2cm]{geometry} % Margenes
\usepackage[T1]{fontenc}                              % Indicamos la codificacion de las fuentes.
\usepackage[utf8x]{inputenc}                          % Definimos la codificacion.
\usepackage{lmodern}                                  % Para poder usar acentos.
\usepackage[spanish]{babel}                           % Usaremos idioma español.
\usepackage{amsmath}                                  % Para formulas matematicas.
\usepackage{graphicx}                                 % Para imagenes.
\usepackage{float}                                    % Para posicionar objetos.
\usepackage{booktabs}                                 % Para formatear tablas.
\usepackage{hyperref}                                 % Para enlaces y referencias.
\usepackage{enumerate} 
%\usepackage{colortbl}
%%%%%%%%%%%%%%%%%%%%%%%%%%%%%%%%%%%%%%%%%%%%%%%%%%%%%%%%%%%%%%%%%%%%%%%%%%%%%%%%%%%%%%%%%

% Los logos tienen posiciones relativas al nombre de la escuela.
% Cada imagen esta desplazada con respecto al texto, en este caso nombre de la univseridad.
% No se necesitan paquetes adicionales, el entorno estandar para imagenes de LaTeX puede hacerlo.
% El truco esta en definir una imagen de tamaño cero, asi no afecta al centrar los titulos.
\def\logoUNAM{%
  \begin{picture}(0,0)\unitlength=1cm
    \put (-3.5,-3) {\includegraphics[width=8em]{images/escudo-unam}}
  \end{picture}
}

\def\logoFI{%
  \begin{picture}(0,0)\unitlength=1cm
    \put (0.5,-3) {\includegraphics[width=8em]{images/escudo-fi}}
  \end{picture}
}

%%%%%%%%%%%%%%%%%%%%%%%%%%%%%%%%%%%%%%%%%%%%%%%%%%%%%%%%%%%%%%%%%%%%%%%%%%%%%%%%%%%%%%%%%

\author{Pérez Navarro Maria Yesica - 414039694}  % Autor de la actividad.
\title{Práctica 12: Modulo de comunicación UART.}                % Titulo de la actividad.
\date{dd/mm/yyyy}                                           % Fecha de entrega.
\def\universidad{Universidad Nacional Autónoma de México}   % Nombre de la universidad.
\def\facultad{Facultad de Ingeniería}                              % Nombre de la facultdad.
\def\semestre{2018-1}                                     % Semestre lectivo.
\def\materia{Lab. Microcontroladores y Microprocesadores - Grupo 03}               % Nombre de la materia y grupo.
\makeatletter

%%%%%%%%%%%%%%%%%%%%%%%%%%%%%%%%%%%%%%%%%%%%%%%%%%%%%%%%%%%%%%%%%%%%%%%%%%%%%%%%%%%%%%%%%

\begin{document}
  
  % Titulo del documento con logos.
  \begin{center}
    \logoUNAM {\Large \universidad} \logoFI\par
    {\large \facultad}\par
    \semestre\par
    \materia\par
    \@author\par
    \@date\par
    \@title
  \end{center}

  \hrulefill\par

  \pagenumbering{gobble}                              % Oculta el numero de pagina.
%  \tableofcontents                                    % Crea el indice o tabla de contenido.

%%%%%%%%%%%%%%%%%%%%%%%%%%%%%%%%%%%%%%%%%%%%%%%%%%%%%%%%%%%%%%%%%%%%%%%%%%%%%%%%%%%%%%%%%

  %\newpage                                            % Inserta una pagina nueva.
  \pagenumbering{arabic}                              % Muestra el numero de pagina.
  
  \section{Seguridad en la Ejecución.}
  \begin{table}[H]
  	\begin{tabular}{|l|l|l|}
  		\hline
  		 & Peligro o fuente de energía & Riesgo asociado  \\ \hline
  		1 & Manejo de Corriente Alterna &Electrochoque    \\ \hline
  		2 & Manejo de corriente Continua & Daño al equipo \\ \hline
  	\end{tabular}
  	\centering
  \end{table}

\section{Objetivos de aprendizaje.}
\begin{itemize}
	\item El alumno aprenderá a programar la unidad UART de la TIVA para realizar enlaces para transferencia de datos usando un puerto serie con protocolo RS232. 
	
\end{itemize}

\section{Material y equipo.}
 
\begin{itemize}
	\item Tarjeta TIVA, 
	\item Software RealTerm
	\item C.I. MAX232
	\item Cables de 1 m o más de largo. 
	\item Protoboard. 
	
\end{itemize}
  

  
\section{Actividad previa.}                   

  \begin{enumerate}[a)]
	\item Leer la guía proporcionada. 
	\item Configurar los registros con los valores faltantes en el código proporcionado. 
	\item ¿Cuál es la señalización del protocolo Serial RS232? ¿Cuáles son las velocidades más comunes para el protocolo RS232? 
	\item ¿Cuál es la duración de un bit si la velocidad de transmisión es de 9600 bits/s? 
	\item ¿Qué es la paridad de un número y el bit de paridad? 
	\item ¿Cuáles son los niveles de voltaje de las líneas RS232 y que circuito sirve para acondicionamiento de los niveles de voltaje a 5 V? 
	
	
\end{enumerate}

\section{Desarrollo}

\begin{enumerate}
	\item Compruebe en clase la configuración de los registros del UART0. 
	\item Pruebe el código fuente empleando la terminal RealTerm, atendiendo las indicaciones y tomando nota de las funcionalidades de la Interfaz. 
	\item  Creación de una conexión Punto a Punto. En un programa nuevo, implemente la configuración adicional de un modulo UART diferente al 0 con acceso a las 
	terminales de la tarjeta (supongamos UART 1). La velocidad del modulo será de 115200 bps, 8,N,1. Implemente la conexión hacia otra tarjeta de otro estudiante. De ser necesario implemente la interfaz de Hardware con el C.I. MAX232. Los datos enviados por el primer estudiante a través de la terminal PC y recibidos por la TIVA, se enviarán por la UART1 hacia otra tarjeta que las recibirá por la UART1 y a su vez, los enviará por la UART0 a la terminal PC del segundo estudiante. Tenga cuidado de hacer una conexión cruzada de las UART1 (Tx1 Rx2, Rx1 Tx2) 
	
\end{enumerate} 



 
  \section{Cuestionario}
   
  
  \begin{enumerate}[a)]

  	\item ¿A qué se refiere la unidad de transferencia: baudios?
  	\item  ¿Cuántas líneas se requieren al menos para implementar una comunicación serial RS232 Full Duplex?
  	\item  ¿Qué tipo de codificación emplean las terminales para el despliegue de datos? 
  	\item ¿Para qué nos sirve el C.I. MAX232? 
  	  
  \end{enumerate}




\section{Conclusiones.}
  \begin{thebibliography}{}                           % Bibliografia
    \bibitem{ref:cita}                                % Etiqueta con la que se hara la referencia o cita.
      Como citar: \url{http://www.cva.itesm.mx/biblioteca/pagina_con_formato_version_oct/apa.htm} % URL de apoyo para citar.

    \bibitem{ref:web1}
      Autor,
      (Fecha de publicacion),
      Titulo, paginas,
      Fecha de recuperacion,
      Sitio web: \url{http://www.google.com}

    \bibitem{ref:github}
      Repositorio del proyecto \url{https://github.com/penserbjorne}
  \end{thebibliography}

%%%%%%%%%%%%%%%%%%%%%%%%%%%%%%%%%%%%%%%%%%%%%%%%%%%%%%%%%%%%%%%%%%%%%%%%%%%%%%%%%%%%%%%%%

\end{document}
