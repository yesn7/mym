%%%%%%%%%%%%%%%%%%%%%%%%%%%%%%%%%%%%%%%%%%%%%%%%%%%%%%%%%%%%%%%%%%%%%%%%%%%%%%%%%%%%%%%%%
% Autor:        Aguilar Enriquez, Paul Sebastian a.k.a. Penserbjorne
% Fecha:        05/02/2017
% Descripcion:  Plantilla base para actividades o tareas.
%%%%%%%%%%%%%%%%%%%%%%%%%%%%%%%%%%%%%%%%%%%%%%%%%%%%%%%%%%%%%%%%%%%%%%%%%%%%%%%%%%%%%%%%%

\documentclass[a4paper,11pt]{article}                 % Papel tamaño carta, texto de 11pt.

\usepackage[top=2cm, bottom=2cm, left=2.2cm, right=2.2cm]{geometry} % Margenes
\usepackage[T1]{fontenc}                              % Indicamos la codificacion de las fuentes.
\usepackage[utf8x]{inputenc}                          % Definimos la codificacion.
\usepackage{lmodern}                                  % Para poder usar acentos.
\usepackage[spanish]{babel}                           % Usaremos idioma español.
\usepackage{amsmath}                                  % Para formulas matematicas.
\usepackage{graphicx}                                 % Para imagenes.
\usepackage{float}                                    % Para posicionar objetos.
\usepackage{booktabs}                                 % Para formatear tablas.
\usepackage{hyperref}                                 % Para enlaces y referencias.
\usepackage{enumerate} 
%\usepackage{colortbl}
%%%%%%%%%%%%%%%%%%%%%%%%%%%%%%%%%%%%%%%%%%%%%%%%%%%%%%%%%%%%%%%%%%%%%%%%%%%%%%%%%%%%%%%%%

% Los logos tienen posiciones relativas al nombre de la escuela.
% Cada imagen esta desplazada con respecto al texto, en este caso nombre de la univseridad.
% No se necesitan paquetes adicionales, el entorno estandar para imagenes de LaTeX puede hacerlo.
% El truco esta en definir una imagen de tamaño cero, asi no afecta al centrar los titulos.
\def\logoUNAM{%
  \begin{picture}(0,0)\unitlength=1cm
    \put (-3.5,-3) {\includegraphics[width=8em]{images/escudo-unam}}
  \end{picture}
}

\def\logoFI{%
  \begin{picture}(0,0)\unitlength=1cm
    \put (0.5,-3) {\includegraphics[width=8em]{images/escudo-fi}}
  \end{picture}
}

%%%%%%%%%%%%%%%%%%%%%%%%%%%%%%%%%%%%%%%%%%%%%%%%%%%%%%%%%%%%%%%%%%%%%%%%%%%%%%%%%%%%%%%%%

\author{Pérez Navarro Maria Yesica - 414039694}  % Autor de la actividad.
\title{Práctica 10: Interrupciones SysTicK y GPIOs.}                % Titulo de la actividad.
\date{dd/mm/yyyy}                                           % Fecha de entrega.
\def\universidad{Universidad Nacional Autónoma de México}   % Nombre de la universidad.
\def\facultad{Facultad de Ingeniería}                              % Nombre de la facultdad.
\def\semestre{2018-1}                                     % Semestre lectivo.
\def\materia{Lab. Microcontroladores y Microprocesadores - Grupo 03}               % Nombre de la materia y grupo.
\makeatletter

%%%%%%%%%%%%%%%%%%%%%%%%%%%%%%%%%%%%%%%%%%%%%%%%%%%%%%%%%%%%%%%%%%%%%%%%%%%%%%%%%%%%%%%%%

\begin{document}
  
  % Titulo del documento con logos.
  \begin{center}
    \logoUNAM {\Large \universidad} \logoFI\par
    {\large \facultad}\par
    \semestre\par
    \materia\par
    \@author\par
    \@date\par
    \@title
  \end{center}

  \hrulefill\par

  \pagenumbering{gobble}                              % Oculta el numero de pagina.
%  \tableofcontents                                    % Crea el indice o tabla de contenido.

%%%%%%%%%%%%%%%%%%%%%%%%%%%%%%%%%%%%%%%%%%%%%%%%%%%%%%%%%%%%%%%%%%%%%%%%%%%%%%%%%%%%%%%%%

  %\newpage                                            % Inserta una pagina nueva.
  \pagenumbering{arabic}                              % Muestra el numero de pagina.
  
  \section{Seguridad en la Ejecución.}
  \begin{table}[H]
  	\begin{tabular}{|l|l|l|}
  		\hline
  		 & Peligro o fuente de energía & Riesgo asociado  \\ \hline
  		1 & Manejo de Corriente Alterna &Electrochoque    \\ \hline
  		2 & Manejo de corriente Continua & Daño al equipo \\ \hline
  	\end{tabular}
  	\centering
  \end{table}

\section{Objetivos de aprendizaje.}
\begin{itemize}
	\item El alumno empleará el ADC del microcontrolador TM4C1294 para capturar señales analógicas proveniente de algún sensor y posteriormente procesarlas con el microcontrolador. 
\end{itemize}

\section{Material y equipo.}
 
\begin{itemize}
	\item Tarjeta de desarrollo y ambiente IDE CCS. 
	\item 8 Leds y resistencias
	\item  1 Potenciómetro
\end{itemize}
  

  
\section{Actividad previa.}                   

  \begin{enumerate}[a)]
	\item ¿Cuántos ADC tiene la tarjeta de desarrollo Tiva TM4C1294?
	\item  ¿Cuál es la resolución en bits de los ADC? 
	\item ¿Qué operación lógica se tiene que hacer para truncar un valor de 12 bits a 8 bits? 
	\item ¿Cuál es la resolución en volts/lsb de los ADC?
	\item ¿Cuál es el Voltaje de referencia del ADC y su rango de entrada?
	\item Para un ADC, ¿qué significa que sea Single-Ended o Diferencial?
	\item ¿Cuántas entradas multiplexadas tiene el ADC del Tiva? 
	\item ¿Cuál es el procedimiento de configuración de las entradas para que funcione con entradas analógicas? 
	\item ¿Cuándo se sabe que un ADC ha terminado la conversión de una muestra?, Para el Tiva, ¿cuándo sabemos que ha terminado una conversión? 
	\item En el ADC del TIVA, ¿qué es un secuenciador y cuantas muestras puede tomar cada uno? 
	
	
\end{enumerate}

\section{Desarrollo}

En el código proporcionado, configurar los registros del sistema y del ADC0 con los valores adecuados para configurar una Terminal de entrada como analógica (PE4), la cual recibe una señal analógica de 0 a 3.3V, y configurar el Puerto K de 8 bits como digital de salida (sentencias 1 a 5a) para desplegar a través de LEDs el resultado de conversión truncado a 8 bits, empleando la técnica de encuesta o polling. 

Configurar el Secuenciador de Muestras 3 (SS3), realizando inicio de conversión por software (o procesador), encender y apagar el PLL para usar como fuente de reloj de conversión el oscilador PIOSC (sentencias 7 a 15). 

Iniciar conversión comenzando por limpiar la bandera de fin de conversión en el registro ISC; preguntar por el fin de la conversión monitorizando la bandera RIS del secuenciador 3. Recuperar los 12 bits de resultado y posteriormente truncarlo para desplegar el resultado en 8 bits (sentencias 16 a 19). 



 
  \section{Cuestionario}
   
  
  \begin{enumerate}[a)]

  	\item ¿Cómo se inicia el muestreo por software en el programa? Si se requiere conversión continua, escriba la sentencia que la configura y el valor correcto. 
  	\item ¿Cuáles registros configuraría y cuál sería su valor, si requiere que la entrada analógica sea la PE0 ?
  	  
  \end{enumerate}




\section{Conclusiones.}
  \begin{thebibliography}{}                           % Bibliografia
    \bibitem{ref:cita}                                % Etiqueta con la que se hara la referencia o cita.
      Como citar: \url{http://www.cva.itesm.mx/biblioteca/pagina_con_formato_version_oct/apa.htm} % URL de apoyo para citar.

    \bibitem{ref:web1}
      Autor,
      (Fecha de publicacion),
      Titulo, paginas,
      Fecha de recuperacion,
      Sitio web: \url{http://www.google.com}

    \bibitem{ref:github}
      Repositorio del proyecto \url{https://github.com/penserbjorne}
  \end{thebibliography}

%%%%%%%%%%%%%%%%%%%%%%%%%%%%%%%%%%%%%%%%%%%%%%%%%%%%%%%%%%%%%%%%%%%%%%%%%%%%%%%%%%%%%%%%%

\end{document}
