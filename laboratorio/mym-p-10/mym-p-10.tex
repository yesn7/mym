%%%%%%%%%%%%%%%%%%%%%%%%%%%%%%%%%%%%%%%%%%%%%%%%%%%%%%%%%%%%%%%%%%%%%%%%%%%%%%%%%%%%%%%%%
% Autor:        Aguilar Enriquez, Paul Sebastian a.k.a. Penserbjorne
% Fecha:        05/02/2017
% Descripcion:  Plantilla base para actividades o tareas.
%%%%%%%%%%%%%%%%%%%%%%%%%%%%%%%%%%%%%%%%%%%%%%%%%%%%%%%%%%%%%%%%%%%%%%%%%%%%%%%%%%%%%%%%%

\documentclass[a4paper,11pt]{article}                 % Papel tamaño carta, texto de 11pt.

\usepackage[top=2cm, bottom=2cm, left=2.2cm, right=2.2cm]{geometry} % Margenes
\usepackage[T1]{fontenc}                              % Indicamos la codificacion de las fuentes.
\usepackage[utf8x]{inputenc}                          % Definimos la codificacion.
\usepackage{lmodern}                                  % Para poder usar acentos.
\usepackage[spanish]{babel}                           % Usaremos idioma español.
\usepackage{amsmath}                                  % Para formulas matematicas.
\usepackage{graphicx}                                 % Para imagenes.
\usepackage{float}                                    % Para posicionar objetos.
\usepackage{booktabs}                                 % Para formatear tablas.
\usepackage{hyperref}                                 % Para enlaces y referencias.
\usepackage{enumerate} 
%\usepackage{colortbl}
%%%%%%%%%%%%%%%%%%%%%%%%%%%%%%%%%%%%%%%%%%%%%%%%%%%%%%%%%%%%%%%%%%%%%%%%%%%%%%%%%%%%%%%%%

% Los logos tienen posiciones relativas al nombre de la escuela.
% Cada imagen esta desplazada con respecto al texto, en este caso nombre de la univseridad.
% No se necesitan paquetes adicionales, el entorno estandar para imagenes de LaTeX puede hacerlo.
% El truco esta en definir una imagen de tamaño cero, asi no afecta al centrar los titulos.
\def\logoUNAM{%
  \begin{picture}(0,0)\unitlength=1cm
    \put (-3.5,-3) {\includegraphics[width=8em]{images/escudo-unam}}
  \end{picture}
}

\def\logoFI{%
  \begin{picture}(0,0)\unitlength=1cm
    \put (0.5,-3) {\includegraphics[width=8em]{images/escudo-fi}}
  \end{picture}
}

%%%%%%%%%%%%%%%%%%%%%%%%%%%%%%%%%%%%%%%%%%%%%%%%%%%%%%%%%%%%%%%%%%%%%%%%%%%%%%%%%%%%%%%%%

\author{Pérez Navarro Maria Yesica - 414039694}  % Autor de la actividad.
\title{Práctica 10: Interrupciones SysTicK y GPIOs.}                % Titulo de la actividad.
\date{dd/mm/yyyy}                                           % Fecha de entrega.
\def\universidad{Universidad Nacional Autónoma de México}   % Nombre de la universidad.
\def\facultad{Facultad de Ingeniería}                              % Nombre de la facultdad.
\def\semestre{2018-1}                                     % Semestre lectivo.
\def\materia{Lab. Microcontroladores y Microprocesadores - Grupo 03}               % Nombre de la materia y grupo.
\makeatletter

%%%%%%%%%%%%%%%%%%%%%%%%%%%%%%%%%%%%%%%%%%%%%%%%%%%%%%%%%%%%%%%%%%%%%%%%%%%%%%%%%%%%%%%%%

\begin{document}
  
  % Titulo del documento con logos.
  \begin{center}
    \logoUNAM {\Large \universidad} \logoFI\par
    {\large \facultad}\par
    \semestre\par
    \materia\par
    \@author\par
    \@date\par
    \@title
  \end{center}

  \hrulefill\par

  \pagenumbering{gobble}                              % Oculta el numero de pagina.
%  \tableofcontents                                    % Crea el indice o tabla de contenido.

%%%%%%%%%%%%%%%%%%%%%%%%%%%%%%%%%%%%%%%%%%%%%%%%%%%%%%%%%%%%%%%%%%%%%%%%%%%%%%%%%%%%%%%%%

  %\newpage                                            % Inserta una pagina nueva.
  \pagenumbering{arabic}                              % Muestra el numero de pagina.
  
  \section{Seguridad en la Ejecución.}
  \begin{table}[H]
  	\begin{tabular}{|l|l|l|}
  		\hline
  		 & Peligro o fuente de energía & Riesgo asociado  \\ \hline
  		1 & Manejo de Corriente Alterna &Electrochoque    \\ \hline
  		2 & Manejo de corriente Continua & Daño al equipo \\ \hline
  	\end{tabular}
  	\centering
  \end{table}

\section{Objetivos de aprendizaje.}
\begin{itemize}
	\item El alumno aplicará la técnica de ejecución de módulos o subrutinas de alta prioridad, síncrona (SysTick) o asíncrona (GPIOs) empleando Interrupciones en el Nucleo ARM. 
\end{itemize}

\section{Material y equipo.}
 
\begin{itemize}
	\item Tarjeta de desarrollo y ambiente IDE CCS. 
	\item 1 Encoder Rotativo (En tarjeta o el dispositivo simple)
	\item Resistencias de 1KOhm
	\item Capacitores de 100nF/0.1uF (Nomenclatura 104)
	\item Leds
\end{itemize}
  

  
\section{Actividad previa.}                   

  \begin{enumerate}[a)]
	\item En un Microcontrolador, Microprocesador, ¿en qué consisten la técnica de Polling (encuesta) y la técnica de interrupción?
	
	\item ¿Qué es un vector de Interrupción, una ISR, y la bandera de Interrupción?.
	
	\item Describa las características del Código Gray.
	
	\item ¿Cómo funciona un encoder y cuál es su conexión eléctrica?
	
	\item ¿Cómo se detecta el sentido de giro en un encoder?
	
	\item Escriba las líneas de código con sus correspondientes comentarios, en un archivo fuente 
	para configurar el bit 0 del Puerto N de la tarjeta TIVA como salida, y el bit 0 del puerto M 
	como  entrada.  Alambre  un  Push-button    normalmente  abierto  a  esta  entrada,  para  que 
	cuando lo presione, entregue un ‘1’ lógico. En consecuencia configure la entrada para que 
	esta acción se lea como un ‘1’ en la terminal PM0.  
	
	
\end{enumerate}
% Insertamos nueva seccion, SI aparece en la tabla de contenido.

\section{Desarrollo}
Uso del Timer SysTick por Polling:
 
 \begin{enumerate}[a)]
 \item Los registros SysTick Control and Status Register (STCTL), SysTick Reload Value Register (STRELOAD) controlan toda la configuración de este Timer. Identifique el nombre de estos registros en el archivo de cabecera tm4c1294ncpdt.h Configure de los registros del SysTick para Modo continuo (Multi-shooting), sin interrupciones y con fuente de reloj PIOSC/4. Considerando que el SysClk = PIOSC = 16MHz, calcule la carga del registro LOAD (N incrementos del contador) para que produzca una cuenta de 1 ms y otra de 500 ms. Habilite la interrupción del SysTick y en la ISR correspondiente cambien el estado lógico del bit PN0. Pruebe ambos valores de carga observando el estado del PN0 en el osciloscopio. 
 	
 \end{enumerate}
 
Uso del Timer SysTick por Interrupción: 
  \begin{enumerate}[b)]
 	\item Habilite en el registro de control del SysTick las Interrupciones. Esta interrupción pasa directamente al NVIC, sin activar registros “Mask” por lo que es una interrupción no mascarable, es decir, que siempre las gestiona el NVIC cuando se activan desde el periférico. 
  \end{enumerate}
 
   \begin{enumerate}[c)]
 	\item En la ISR del SysTick,  use una variable para llevar la cuenta del número de veces que esta ISR se ejecuta. 
 	
 \end{enumerate}

Uso del Encoder por Interrupción 
    \begin{enumerate}[d)]
 	\item  Conecte el Encoder ahora en el Puerto M (Pines M1 y M0), configure como entrada y dependiendo del hardware, configure o no resistencias de Pull up/down. Agregue los capacitores como antirebote para las líneas del encoder.  Habilite  la Interrupción de los pines M0 y M1 en el registro Interrupt Mask (IM). Habilite la gestión de interrupciones del Puerto M  en el NVIC, identificando en cuál registro de habilitación $NVIC_ENx$ se ubica el bit de la interrupción del Puerto M; el número de Interrupción del puerto M es el número:       Configure la generación de interrupción en transiciones de ambos niveles (Edges). 
 	
 	En la ISR del puerto M, use una variable para llevar la cuenta del número de transiciones que ocurren en las líneas del encoder.  Al mismo tiempo, escriba una 
 	rutina que determine el sentido de giro del encoder; considere el estado anterior y el estado actual. 
 	
 	
 \end{enumerate}
 
  \section{Cuestionario}
   
  
  \begin{enumerate}[a)]

  	\item ¿De qué forma se limpia la bandera de ocurrencia de interrupción para el SysTick? 
  	\item ¿Cuál es la instrucción en ensamblador para habilitar y cual es para deshabilitar todas las Interrupciones?
  	\item ¿En qué registros se puede monitorizar una bandera del estado de un evento que puede disparar una Interrupción? 
  	\item ¿En qué registros se habilita una interrupción Raw para su manejo por el NVIC?
  	\item ¿En qué registros se puede monitorizar una bandera del estado de una interrupción que se pasa al NVIC?
  	\item ¿En qué registros se limpian las banderas indicadoras de la ocurrencia de Interrupción para? 
  	  
  \end{enumerate}




\section{Conclusiones.}
  \begin{thebibliography}{}                           % Bibliografia
    \bibitem{ref:cita}                                % Etiqueta con la que se hara la referencia o cita.
      Como citar: \url{http://www.cva.itesm.mx/biblioteca/pagina_con_formato_version_oct/apa.htm} % URL de apoyo para citar.

    \bibitem{ref:web1}
      Autor,
      (Fecha de publicacion),
      Titulo, paginas,
      Fecha de recuperacion,
      Sitio web: \url{http://www.google.com}

    \bibitem{ref:github}
      Repositorio del proyecto \url{https://github.com/penserbjorne}
  \end{thebibliography}

%%%%%%%%%%%%%%%%%%%%%%%%%%%%%%%%%%%%%%%%%%%%%%%%%%%%%%%%%%%%%%%%%%%%%%%%%%%%%%%%%%%%%%%%%

\end{document}
