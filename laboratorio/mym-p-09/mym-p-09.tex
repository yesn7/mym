%%%%%%%%%%%%%%%%%%%%%%%%%%%%%%%%%%%%%%%%%%%%%%%%%%%%%%%%%%%%%%%%%%%%%%%%%%%%%%%%%%%%%%%%%
% Autor:        Aguilar Enriquez, Paul Sebastian a.k.a. Penserbjorne
% Fecha:        05/02/2017
% Descripcion:  Plantilla base para actividades o tareas.
%%%%%%%%%%%%%%%%%%%%%%%%%%%%%%%%%%%%%%%%%%%%%%%%%%%%%%%%%%%%%%%%%%%%%%%%%%%%%%%%%%%%%%%%%

\documentclass[a4paper,11pt]{article}                 % Papel tamaño carta, texto de 11pt.

\usepackage[top=2cm, bottom=2cm, left=2.2cm, right=2.2cm]{geometry} % Margenes
\usepackage[T1]{fontenc}                              % Indicamos la codificacion de las fuentes.
\usepackage[utf8x]{inputenc}                          % Definimos la codificacion.
\usepackage{lmodern}                                  % Para poder usar acentos.
\usepackage[spanish]{babel}                           % Usaremos idioma español.
\usepackage{amsmath}                                  % Para formulas matematicas.
\usepackage{graphicx}                                 % Para imagenes.
\usepackage{float}                                    % Para posicionar objetos.
\usepackage{booktabs}                                 % Para formatear tablas.
\usepackage{hyperref}                                 % Para enlaces y referencias.
\usepackage{enumerate} 
%\usepackage{colortbl}
%%%%%%%%%%%%%%%%%%%%%%%%%%%%%%%%%%%%%%%%%%%%%%%%%%%%%%%%%%%%%%%%%%%%%%%%%%%%%%%%%%%%%%%%%

% Los logos tienen posiciones relativas al nombre de la escuela.
% Cada imagen esta desplazada con respecto al texto, en este caso nombre de la univseridad.
% No se necesitan paquetes adicionales, el entorno estandar para imagenes de LaTeX puede hacerlo.
% El truco esta en definir una imagen de tamaño cero, asi no afecta al centrar los titulos.
\def\logoUNAM{%
  \begin{picture}(0,0)\unitlength=1cm
    \put (-3.5,-3) {\includegraphics[width=8em]{images/escudo-unam}}
  \end{picture}
}

\def\logoFI{%
  \begin{picture}(0,0)\unitlength=1cm
    \put (0.5,-3) {\includegraphics[width=8em]{images/escudo-fi}}
  \end{picture}
}

%%%%%%%%%%%%%%%%%%%%%%%%%%%%%%%%%%%%%%%%%%%%%%%%%%%%%%%%%%%%%%%%%%%%%%%%%%%%%%%%%%%%%%%%%

\author{Pérez Navarro Maria Yesica - 414039694}  % Autor de la actividad.
\title{Práctica 09: Puertos de entrada/salida 3.}                % Titulo de la actividad.
\date{dd/mm/yyyy}                                           % Fecha de entrega.
\def\universidad{Universidad Nacional Autónoma de México}   % Nombre de la universidad.
\def\facultad{Facultad de Ingeniería}                              % Nombre de la facultdad.
\def\semestre{2018-1}                                     % Semestre lectivo.
\def\materia{Lab. Microcontroladores y Microprocesadores - Grupo 03}               % Nombre de la materia y grupo.
\makeatletter

%%%%%%%%%%%%%%%%%%%%%%%%%%%%%%%%%%%%%%%%%%%%%%%%%%%%%%%%%%%%%%%%%%%%%%%%%%%%%%%%%%%%%%%%%

\begin{document}
  
  % Titulo del documento con logos.
  \begin{center}
    \logoUNAM {\Large \universidad} \logoFI\par
    {\large \facultad}\par
    \semestre\par
    \materia\par
    \@author\par
    \@date\par
    \@title
  \end{center}

  \hrulefill\par

  \pagenumbering{gobble}                              % Oculta el numero de pagina.
%  \tableofcontents                                    % Crea el indice o tabla de contenido.

%%%%%%%%%%%%%%%%%%%%%%%%%%%%%%%%%%%%%%%%%%%%%%%%%%%%%%%%%%%%%%%%%%%%%%%%%%%%%%%%%%%%%%%%%

  %\newpage                                            % Inserta una pagina nueva.
  \pagenumbering{arabic}                              % Muestra el numero de pagina.
  
  \section{Seguridad en la Ejecución.}
  \begin{table}[H]
  	\begin{tabular}{|l|l|l|}
  		\hline
  		 & Peligro o fuente de energía & Riesgo asociado  \\ \hline
  		1 & Manejo de Corriente Alterna &Electrochoque    \\ \hline
  		2 & Manejo de corriente Continua & Daño al equipo \\ \hline
  	\end{tabular}
  	\centering
  \end{table}

\section{Objetivos de aprendizaje.}
\begin{itemize}
	\item El alumno programará los Puertos de Entrada/Salida del procesador ARM M4. 
\end{itemize}

\section{Material y equipo.}
 
\begin{itemize}
	\item Hojas de datos de LCD 16x2 genérico 
	\item Programas proporcionados. 
	\item Sistema de desarrollo. 
	\item Display LCD 16x2. 
	\item Teclado Matricial 4x4.
	\item Cables jumper. 
	\item Headers macho o hembra de dos líneas para soldar a la tarjeta de desarrollo y acceder a los pines que no disponen de terminales de conexión. 
	\item LEDs y resistencias para desplegar datos de salida. 
	\item Presentar el teclado y LCD ya conectados al sistema de desarrollo.   
\end{itemize}
  

  
\section{Actividad previa.}                   

  \begin{enumerate}[a)]
	\item  Soldar las terminales Header para acceder a las terminales.  
	
	\item  Teclado Matricial 4x4. 
	
	Estudie el funcionamiento de un teclado matricial de 4x4 junto con el código proporcionado, consistente en 3 archivos que deberá incluir en un proyecto.  
	
	Considere que el escaneo del teclado se realiza activando las líneas de los renglones y leyendo el puerto donde se conectan las columnas. 
	
	Considere que las líneas de entrada para las columnas deben tener un estado definido, por lo que se recomienda activar las resistencias de Pull-down internas en caso de no implementarlas externamente. 
	
	Implemente el circuito para el teclado matricial. Considere que las teclas A,B,C,D,$*$ y \# corresponden a los números 10-15 cuando se use la identificación de la tecla presionada y su despliegue en 4 Leds.
	
	\item  Display LCD de 16 caracteres x 2 Líneas. 
	
	Estudie el funcionamiento del LCD con el material proporcionado. Resuma en un diagrama de Flujo la secuencia de pasos empleada para inicializar el LCD que se emplea en el programa proporcionado. Implemente las conexiones para el LCD (5 o 3 V). 
	
	
\end{enumerate}


\section{Desarrollo}
 Teclado Matricial 4x4. 
 
 \begin{enumerate}[a)]
 \item Con el Programa fuente proporcionado, estudie su funcionamiento y agregue la  configuración de los puertos que se emplean para el escaneo del teclado. 
 
 Use el Puerto M (bits 3:0) como el controlador de los renglones y el Puerto H (bits 3:0) para leer las columnas.  
 
 Use el Puerto N (bits 3:0) para desplegar en LEDs el valor de la tecla presionada. 
 
 \item  Visualice en la sesión de Debug todas las variables que pueden cambiar durante la ejecución normal del programa. Note que la variable $Key_Pressed$ corresponde al carácter mapeado en el arreglo KeyboardTable, por lo que en la sesión de Debug, podrá apreciar esta variable como una variable tipo char y codificada como ASCII. 
\end{enumerate}
 
 Display LCD de 16 caracteres x 2 Líneas. 
 
  \begin{enumerate}[a)]
 	\item  Con el Programa fuente proporcionado, estudie su funcionamiento y agregue la  configuración de los puertos que se emplean para el manejo del LCD.  
 	
 	Use el Puerto K(bits7:0) para el bus de datos del LCD y el Puerto L(bits 3:0) para las líneas de control.  Se proporcionan las rutinas para escribir un comando al LCD y para escribir caracteres.  
 	
 	
 	\item  En el programa no se lee el Bit de BUSY del LCD, por lo que se considera un retardo después de cada escritura que asegure su funcionamiento. Mejore el programa escribiendo una rutina que lea este Bit. Considere que se tiene que reconfigurar el puerto de datos como de entrada para leer este bit y regresarlo a puerto de salida cuando ya no se requiera leer el bit BUSY.
 	
 	\item  Defina como mensaje a mostrar su nombre completo y despliéguelo en el LCD empleando desplazamiento de la pantalla a la izquierda.  
 	
 	\item  Visualice en la sesión de Debug todas las variables que pueden cambiar durante la ejecución normal del programa. 
 	
 \end{enumerate}
 
 
  \section{Cuestionario}
  
  Teclado matricial. 
  
  \begin{enumerate}[a)]

  	\item  Explique lo que pasa cuando comenta las siguientes líneas en el código del Teclado Matricial.    
  	
    \item  Explique la función que cumple cada sentencia por separado. 
    
    \item  Explique para qué sirve la declaración de la variable: 
  	  
  \end{enumerate}

Display LCD. 

  \begin{enumerate}[a)]
	\item El programa proporcionado escribe datos al LCD por medio de 8 bits. Otra forma de hacerlo es con 4 bits, Primero configurando el LCD para este modo de trabajo y después enviando la parte alta del dato y después la parte baja.
	 
	   Escriba la rutina que controle el LCD con 4 bits para el bus de datos. 
	
\end{enumerate}



\section{Conclusiones.}
  \begin{thebibliography}{}                           % Bibliografia
    \bibitem{ref:cita}                                % Etiqueta con la que se hara la referencia o cita.
      Como citar: \url{http://www.cva.itesm.mx/biblioteca/pagina_con_formato_version_oct/apa.htm} % URL de apoyo para citar.

    \bibitem{ref:web1}
      Autor,
      (Fecha de publicacion),
      Titulo, paginas,
      Fecha de recuperacion,
      Sitio web: \url{http://www.google.com}

    \bibitem{ref:github}
      Repositorio del proyecto \url{https://github.com/penserbjorne}
  \end{thebibliography}

%%%%%%%%%%%%%%%%%%%%%%%%%%%%%%%%%%%%%%%%%%%%%%%%%%%%%%%%%%%%%%%%%%%%%%%%%%%%%%%%%%%%%%%%%

\end{document}
