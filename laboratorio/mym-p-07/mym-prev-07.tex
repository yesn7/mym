%%%%%%%%%%%%%%%%%%%%%%%%%%%%%%%%%%%%%%%%%%%%%%%%%%%%%%%%%%%%%%%%%%%%%%%%%%%%%%%%%%%%%%%%%
% Autor:        Aguilar Enriquez, Paul Sebastian a.k.a. Penserbjorne
% Fecha:        05/02/2017
% Descripcion:  Plantilla base para actividades o tareas.
%%%%%%%%%%%%%%%%%%%%%%%%%%%%%%%%%%%%%%%%%%%%%%%%%%%%%%%%%%%%%%%%%%%%%%%%%%%%%%%%%%%%%%%%%

\documentclass[a4paper,11pt]{article}                 % Papel tamaño carta, texto de 11pt.

\usepackage[top=2cm, bottom=2cm, left=2.2cm, right=2.2cm]{geometry} % Margenes
\usepackage[T1]{fontenc}                              % Indicamos la codificacion de las fuentes.
\usepackage[utf8x]{inputenc}                          % Definimos la codificacion.
\usepackage{lmodern}                                  % Para poder usar acentos.
\usepackage[spanish]{babel}                           % Usaremos idioma español.
\usepackage{amsmath}                                  % Para formulas matematicas.
\usepackage{graphicx}                                 % Para imagenes.
\usepackage{float}                                    % Para posicionar objetos.
\usepackage{booktabs}                                 % Para formatear tablas.
\usepackage{hyperref}                                 % Para enlaces y referencias.
\usepackage{enumerate} 
%\usepackage{colortbl}
%%%%%%%%%%%%%%%%%%%%%%%%%%%%%%%%%%%%%%%%%%%%%%%%%%%%%%%%%%%%%%%%%%%%%%%%%%%%%%%%%%%%%%%%%

% Los logos tienen posiciones relativas al nombre de la escuela.
% Cada imagen esta desplazada con respecto al texto, en este caso nombre de la univseridad.
% No se necesitan paquetes adicionales, el entorno estandar para imagenes de LaTeX puede hacerlo.
% El truco esta en definir una imagen de tamaño cero, asi no afecta al centrar los titulos.
\def\logoUNAM{%
  \begin{picture}(0,0)\unitlength=1cm
    \put (-3.5,-3) {\includegraphics[width=8em]{images/escudo-unam}}
  \end{picture}
}

\def\logoFI{%
  \begin{picture}(0,0)\unitlength=1cm
    \put (0.5,-3) {\includegraphics[width=8em]{images/escudo-fi}}
  \end{picture}
}

%%%%%%%%%%%%%%%%%%%%%%%%%%%%%%%%%%%%%%%%%%%%%%%%%%%%%%%%%%%%%%%%%%%%%%%%%%%%%%%%%%%%%%%%%

\author{Pérez Navarro Maria Yesica - 414039694}  % Autor de la actividad.
\title{Previo 07: Puertos de entrada/salida.}                % Titulo de la actividad.
\date{dd/mm/yyyy}                                           % Fecha de entrega.
\def\universidad{Universidad Nacional Autónoma de México}   % Nombre de la universidad.
\def\facultad{Facultad de Ingeniería}                              % Nombre de la facultdad.
\def\semestre{2018-1}                                     % Semestre lectivo.
\def\materia{Lab. Microcontroladores y Microprocesadores - Grupo 03}               % Nombre de la materia y grupo.
\makeatletter

%%%%%%%%%%%%%%%%%%%%%%%%%%%%%%%%%%%%%%%%%%%%%%%%%%%%%%%%%%%%%%%%%%%%%%%%%%%%%%%%%%%%%%%%%

\begin{document}
  
  % Titulo del documento con logos.
  \begin{center}
    \logoUNAM {\Large \universidad} \logoFI\par
    {\large \facultad}\par
    \semestre\par
    \materia\par
    \@author\par
    \@date\par
    \@title
  \end{center}

  \hrulefill\par

  \pagenumbering{gobble}                              % Oculta el numero de pagina.
%  \tableofcontents                                    % Crea el indice o tabla de contenido.

%%%%%%%%%%%%%%%%%%%%%%%%%%%%%%%%%%%%%%%%%%%%%%%%%%%%%%%%%%%%%%%%%%%%%%%%%%%%%%%%%%%%%%%%%

  %\newpage                                            % Inserta una pagina nueva.
  \pagenumbering{arabic}                              % Muestra el numero de pagina.
  
  \section{Actividad previa.}                   
  
  \begin{enumerate}[a)]
  	\item Presente un diagrama de conexión de una Resistencia de Pull-up y Pull-down y explique su funcionamiento.
  	
  	\item  Describa la función de los siguientes registros de los puertos de E/S y su configuración tras un Reset. Indique la dirección de cada registro en el mapa de memoria. 
  	
  	\begin{table}[H]
  		\centering
  		\begin{tabular}{|c|l|l|}
  			\hline
  			Registro                                                                   & Función & \begin{tabular}[c]{@{}l@{}}Estado\\ en reset\end{tabular} \\ \hline
  			\begin{tabular}[c]{@{}c@{}}GPIO DATA\\ \\ Dir: \_\_\_\_\_\_\_\end{tabular} &         &                                                           \\ \hline
  			\begin{tabular}[c]{@{}c@{}}GPIO DIR \\ \\ Dir: \_\_\_\_\_\_\_\end{tabular} &         &                                                           \\ \hline
  			\begin{tabular}[c]{@{}c@{}}GPIOAFSEL\\ Dir: \_\_\_\_\_\_\_\end{tabular}    &         &                                                           \\ \hline
  			\begin{tabular}[c]{@{}c@{}}GPIOPUR\\ \\ Dir: \_\_\_\_\_\_\_\end{tabular}   &         &                                                           \\ \hline
  			\begin{tabular}[c]{@{}c@{}}GPIOPDR\\ Dir: \_\_\_\_\_\_\_\end{tabular}      &         &                                                           \\ \hline
  			\begin{tabular}[c]{@{}c@{}}GPIODEN\\ Dir: \_\_\_\_\_\_\_\end{tabular}      &         &                                                           \\ \hline
  			\begin{tabular}[c]{@{}c@{}}GPIOLOCK\\ Dir: \_\_\_\_\_\_\_\end{tabular}     &         &                                                           \\ \hline
  			\begin{tabular}[c]{@{}c@{}}GPIOPCTL\\ Dir: \_\_\_\_\_\_\_\end{tabular}     &         &                                                           \\ \hline
  		\end{tabular}
  	\end{table}
  	
  	\item Describa la secuencia de pasos secuenciales para programar un puerto paralelo como digital (entrada y salida). 
  	
  	\item Describa el modo de funcionamiento de bits direccionables (o direccionamiento de bits específico) en un puerto GPIO.
  	
  	\item Para hacer lectura y escritura de todos los bits de un puerto, ¿cuál es el registro al cual se hace acceso de lectura/escritura? 
  	
  	\item ¿Cómo se hace referencia a una dirección de un registro en lenguaje C?  
  	
  	\item Revise (y en medida de lo posible ejecute en la tarjeta de desarrollo) los programas blink.asm, simpleIO.asm, main.c y main2.c  considerando el hardware de la tarjeta Tiva TM4C1293. 
  	\\
  	Explique detalladamente lo que hace cada programa. NOTA: para los programas blink.asm, simpleIO.asm incluya los archivos $macros.s$ y $gpio_regs.s$ en la carpeta del proyecto. 
  	\\ 
  	Explique detalladamente lo que hace cada programa. 
  	
  \end{enumerate}
                    % Insertamos nueva seccion, SI aparece en la tabla de contenido.
 
  
  \begin{thebibliography}{}                           % Bibliografia
    \bibitem{ref:cita}                                % Etiqueta con la que se hara la referencia o cita.
      Como citar: \url{http://www.cva.itesm.mx/biblioteca/pagina_con_formato_version_oct/apa.htm} % URL de apoyo para citar.

    \bibitem{ref:web1}
      Autor,
      (Fecha de publicacion),
      Titulo, paginas,
      Fecha de recuperacion,
      Sitio web: \url{http://www.google.com}

    \bibitem{ref:github}
      Repositorio del proyecto \url{https://github.com/penserbjorne}
  \end{thebibliography}

%%%%%%%%%%%%%%%%%%%%%%%%%%%%%%%%%%%%%%%%%%%%%%%%%%%%%%%%%%%%%%%%%%%%%%%%%%%%%%%%%%%%%%%%%

\end{document}
