%%%%%%%%%%%%%%%%%%%%%%%%%%%%%%%%%%%%%%%%%%%%%%%%%%%%%%%%%%%%%%%%%%%%%%%%%%%%%%%%%%%%%%%%%
% Autor:        Aguilar Enriquez, Paul Sebastian a.k.a. Penserbjorne
% Fecha:        05/02/2017
% Descripcion:  Plantilla base para actividades o tareas.
%%%%%%%%%%%%%%%%%%%%%%%%%%%%%%%%%%%%%%%%%%%%%%%%%%%%%%%%%%%%%%%%%%%%%%%%%%%%%%%%%%%%%%%%%

\documentclass[a4paper,11pt]{article}                 % Papel tamaño carta, texto de 11pt.

\usepackage[top=2cm, bottom=2cm, left=2.2cm, right=2.2cm]{geometry} % Margenes
\usepackage[T1]{fontenc}                              % Indicamos la codificacion de las fuentes.
\usepackage[utf8x]{inputenc}                          % Definimos la codificacion.
\usepackage{lmodern}                                  % Para poder usar acentos.
\usepackage[spanish]{babel}                           % Usaremos idioma español.
\usepackage{amsmath}                                  % Para formulas matematicas.
\usepackage{graphicx}                                 % Para imagenes.
\usepackage{float}                                    % Para posicionar objetos.
\usepackage{booktabs}                                 % Para formatear tablas.
\usepackage{hyperref}                                 % Para enlaces y referencias.
\usepackage{enumerate} 
%\usepackage{colortbl}
%%%%%%%%%%%%%%%%%%%%%%%%%%%%%%%%%%%%%%%%%%%%%%%%%%%%%%%%%%%%%%%%%%%%%%%%%%%%%%%%%%%%%%%%%

% Los logos tienen posiciones relativas al nombre de la escuela.
% Cada imagen esta desplazada con respecto al texto, en este caso nombre de la univseridad.
% No se necesitan paquetes adicionales, el entorno estandar para imagenes de LaTeX puede hacerlo.
% El truco esta en definir una imagen de tamaño cero, asi no afecta al centrar los titulos.
\def\logoUNAM{%
  \begin{picture}(0,0)\unitlength=1cm
    \put (-3.5,-3) {\includegraphics[width=8em]{images/escudo-unam}}
  \end{picture}
}

\def\logoFI{%
  \begin{picture}(0,0)\unitlength=1cm
    \put (0.5,-3) {\includegraphics[width=8em]{images/escudo-fi}}
  \end{picture}
}

%%%%%%%%%%%%%%%%%%%%%%%%%%%%%%%%%%%%%%%%%%%%%%%%%%%%%%%%%%%%%%%%%%%%%%%%%%%%%%%%%%%%%%%%%

\author{Pérez Navarro Maria Yesica - 414039694}  % Autor de la actividad.
\title{Previo 05: Programación en lenguaje Assembly \\ (Ensamblador) 2}                % Titulo de la actividad.
\date{dd/mm/yyyy}                                           % Fecha de entrega.
\def\universidad{Universidad Nacional Autónoma de México}   % Nombre de la universidad.
\def\facultad{Facultad de Ingeniería}                              % Nombre de la facultdad.
\def\semestre{2018-1}                                     % Semestre lectivo.
\def\materia{Lab. Microcontroladores y Microprocesadores - Grupo 03}               % Nombre de la materia y grupo.
\makeatletter

%%%%%%%%%%%%%%%%%%%%%%%%%%%%%%%%%%%%%%%%%%%%%%%%%%%%%%%%%%%%%%%%%%%%%%%%%%%%%%%%%%%%%%%%%

\begin{document}
  
  % Titulo del documento con logos.
  \begin{center}
    \logoUNAM {\Large \universidad} \logoFI\par
    {\large \facultad}\par
    \semestre\par
    \materia\par
    \@author\par
    \@date\par
    \@title
  \end{center}

  \hrulefill\par

  \pagenumbering{gobble}                              % Oculta el numero de pagina.
%  \tableofcontents                                    % Crea el indice o tabla de contenido.

%%%%%%%%%%%%%%%%%%%%%%%%%%%%%%%%%%%%%%%%%%%%%%%%%%%%%%%%%%%%%%%%%%%%%%%%%%%%%%%%%%%%%%%%%

  %\newpage                                            % Inserta una pagina nueva.
  \pagenumbering{arabic}                              % Muestra el numero de pagina.
  
  \section{Actividad previa.}                   
  
  \begin{enumerate}[a)]
  	\item ¿Cómo se obtiene la representación binaria de un número negativo en el formato de complemento a 2? 
  	
  	\item  ¿Cuál es la longitud en bytes del resultado de una multiplicación no signada (unsigned multiplication) al multiplicar: a) dos valores de 16 bits cada uno? b) dos valores de 32 bits cada uno? 
  	
  	\item  ¿Qué hace el salto condicional “BEQ etiqueta1” ? 
  	
  	\item  ¿Qué hace el salto “BL etiqueta2” ? 
  	
  	\item  Para cada uno de los dos saltos anteriores, ¿se debe regresar al lugar del salto?, ¿cómo?
  	 
  	\item Examine el comportamiento de los segmentos del código proporcionado. Con ayuda de la calculadora, compruebe el resultado de las operaciones.
  	
  \end{enumerate}
                    % Insertamos nueva seccion, SI aparece en la tabla de contenido.
 
  
  \begin{thebibliography}{}                           % Bibliografia
    \bibitem{ref:cita}                                % Etiqueta con la que se hara la referencia o cita.
      Como citar: \url{http://www.cva.itesm.mx/biblioteca/pagina_con_formato_version_oct/apa.htm} % URL de apoyo para citar.

    \bibitem{ref:web1}
      Autor,
      (Fecha de publicacion),
      Titulo, paginas,
      Fecha de recuperacion,
      Sitio web: \url{http://www.google.com}

    \bibitem{ref:github}
      Repositorio del proyecto \url{https://github.com/penserbjorne}
  \end{thebibliography}

%%%%%%%%%%%%%%%%%%%%%%%%%%%%%%%%%%%%%%%%%%%%%%%%%%%%%%%%%%%%%%%%%%%%%%%%%%%%%%%%%%%%%%%%%

\end{document}
