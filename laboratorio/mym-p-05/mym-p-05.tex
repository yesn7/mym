%%%%%%%%%%%%%%%%%%%%%%%%%%%%%%%%%%%%%%%%%%%%%%%%%%%%%%%%%%%%%%%%%%%%%%%%%%%%%%%%%%%%%%%%%
% Autor:        Aguilar Enriquez, Paul Sebastian a.k.a. Penserbjorne
% Fecha:        05/02/2017
% Descripcion:  Plantilla base para actividades o tareas.
%%%%%%%%%%%%%%%%%%%%%%%%%%%%%%%%%%%%%%%%%%%%%%%%%%%%%%%%%%%%%%%%%%%%%%%%%%%%%%%%%%%%%%%%%

\documentclass[a4paper,11pt]{article}                 % Papel tamaño carta, texto de 11pt.

\usepackage[top=2cm, bottom=2cm, left=2.2cm, right=2.2cm]{geometry} % Margenes
\usepackage[T1]{fontenc}                              % Indicamos la codificacion de las fuentes.
\usepackage[utf8x]{inputenc}                          % Definimos la codificacion.
\usepackage{lmodern}                                  % Para poder usar acentos.
\usepackage[spanish]{babel}                           % Usaremos idioma español.
\usepackage{amsmath}                                  % Para formulas matematicas.
\usepackage{graphicx}                                 % Para imagenes.
\usepackage{float}                                    % Para posicionar objetos.
\usepackage{booktabs}                                 % Para formatear tablas.
\usepackage{hyperref}                                 % Para enlaces y referencias.
\usepackage{enumerate} 
\usepackage{verbatim}
\usepackage{fancyvrb}
\usepackage{colortbl}
%%%%%%%%%%%%%%%%%%%%%%%%%%%%%%%%%%%%%%%%%%%%%%%%%%%%%%%%%%%%%%%%%%%%%%%%%%%%%%%%%%%%%%%%%

% Los logos tienen posiciones relativas al nombre de la escuela.
% Cada imagen esta desplazada con respecto al texto, en este caso nombre de la univseridad.
% No se necesitan paquetes adicionales, el entorno estandar para imagenes de LaTeX puede hacerlo.
% El truco esta en definir una imagen de tamaño cero, asi no afecta al centrar los titulos.
\def\logoUNAM{%
  \begin{picture}(0,0)\unitlength=1cm
    \put (-3.5,-3) {\includegraphics[width=8em]{images/escudo-unam}}
  \end{picture}
}

\def\logoFI{%
  \begin{picture}(0,0)\unitlength=1cm
    \put (0.5,-3) {\includegraphics[width=8em]{images/escudo-fi}}
  \end{picture}
}

%%%%%%%%%%%%%%%%%%%%%%%%%%%%%%%%%%%%%%%%%%%%%%%%%%%%%%%%%%%%%%%%%%%%%%%%%%%%%%%%%%%%%%%%%

\author{Pérez Navarro Maria Yesica - 414039694}  % Autor de la actividad.
\title{Práctica 05: Programación en lenguaje\\ Assembly (Ensamblador) 2}                % Titulo de la actividad.
\date{dd/mm/yyyy}                                           % Fecha de entrega.
\def\universidad{Universidad Nacional Autónoma de México}   % Nombre de la universidad.
\def\facultad{Facultad de Ingeniería}                              % Nombre de la facultdad.
\def\semestre{2018-1}                                     % Semestre lectivo.
\def\materia{Lab. Microcontroladores y Microprocesadores - Grupo 03}               % Nombre de la materia y grupo.
\makeatletter

%%%%%%%%%%%%%%%%%%%%%%%%%%%%%%%%%%%%%%%%%%%%%%%%%%%%%%%%%%%%%%%%%%%%%%%%%%%%%%%%%%%%%%%%%

\begin{document}
  
  % Titulo del documento con logos.
  \begin{center}
    \logoUNAM {\Large \universidad} \logoFI\par
    {\large \facultad}\par
    \semestre\par
    \materia\par
    \@author\par
    \@date\par
    \@title
  \end{center}

  \hrulefill\par

  \pagenumbering{gobble}                              % Oculta el numero de pagina.
%  \tableofcontents                                    % Crea el indice o tabla de contenido.

%%%%%%%%%%%%%%%%%%%%%%%%%%%%%%%%%%%%%%%%%%%%%%%%%%%%%%%%%%%%%%%%%%%%%%%%%%%%%%%%%%%%%%%%%

  %\newpage                                            % Inserta una pagina nueva.
  \pagenumbering{arabic}                              % Muestra el numero de pagina.
  
  \section{Seguridad en la Ejecución.}
  \begin{table}[H]
  	\begin{tabular}{|l|l|l|}
  		\hline
  		 & Peligro o fuente de energía & Riesgo asociado  \\ \hline
  		1 & Manejo de Corriente Alterna &Electrochoque    \\ \hline
  		2 & Manejo de corriente Continua & Daño al equipo \\ \hline
  	\end{tabular}
  	\centering
  \end{table}

\section{Objetivos de aprendizaje.}
\begin{itemize}
	\item El alumno aprenderá a utilizar las instrucciones de operaciones aritméticas y lógicas, así como emplear saltos condicionales y llamados a subrutinas. 
\end{itemize}

\section{Material y equipo.}
 
\begin{itemize}
	\item Calculadora en modo “Programador”.
	\item Tarjeta de desarrollo.
	\item CCS IDE. 
\end{itemize}
  

  
\section{Actividad previa.}                   

\begin{enumerate}[a)]
	\item ¿Cómo se obtiene la representación binaria de un número negativo en el formato de complemento a 2? 
	\\ \textsf{Se niega el número y se le suma un uno.}
	
	\item  ¿Cuál es la longitud en bytes del resultado de una multiplicación no signada (unsigned multiplication) al multiplicar: 
	\begin{itemize}
		\item  dos valores de 16 bits cada uno?
		\\ \textsf{El resultado tiene una longitud de 32 bits.}
		\item  dos valores de 32 bits cada uno? 
		\\ \textsf{El resultado tiene una longitud de 64 bits.}
	\end{itemize}
	
	
	\item  ¿Qué hace el salto condicional “BEQ etiqueta1” ? 
	\\ \textsf{Salta hacia la “etiqueta1” si el resultado de una comparación u operación aritmética es igual a 0 (la bandera Z es igual a 0).}
	\item  ¿Qué hace el salto “BL etiqueta2” ? 
	\\ \textsf{Salta hacia una subrutina(“etiqueta2”) y guarda una dirección de retorno en un registro.}
	
	\item  Para cada uno de los dos saltos anteriores, ¿se debe regresar al lugar del salto?, ¿cómo?
	\\ \textsf{Sólo para el salto BL, para regresar a la direccion de retorno se hace con la instruccion BX.}
	
	\item Examine el comportamiento de los segmentos del código proporcionado. Con ayuda de la calculadora, compruebe el resultado de las operaciones.
	
	\newcommand\codeHighlightM[1]{\textcolor[rgb]{.80,0,.60}{\textbf{#1}}}
\newcommand\codeHighlightV[1]{\textcolor[rgb]{.2,.8,0}{\textbf{#1}}}

	\begin{Verbatim}[commandchars=\\\{\}]
	
            \codeHighlightM{.data}
	    .retain       \codeHighlightV{ ;mantiene en memoria variables o símbolos}
                           \codeHighlightV{;no referenciados en programa}
\codeHighlightV{;******** Seccion de datos no inicializados *****************}
            \codeHighlightM{.bss} lista1, 10

\codeHighlightV{;******** Seccion de datos inicializados *****************}
A           \codeHighlightM{.word} 12, 21, 32, 42, 21, 74, 88, 91, 3, 121
            \codeHighlightM{.byte} 12, 21, 32, 42, 21, 74, 88, 91, 3, 121 

\codeHighlightV{;********** **************************************************}
            \codeHighlightM{.global} main
                           \codeHighlightM{.text}
                           \codeHighlightM{.align} 4


\codeHighlightM{main:}
MOV R0, #5
; para decrementar un Registro.
SUBS R0, R0,#5
;;;;;;;;;;;;;;;;;;;;;;;;;;;;;;;;;;;;;;;;;;;;;;;;;;;;;;;;;;;;;;;;;;;;;;;
;;;   Operación resta
; instruccion substraccion con efecto en banderas
; revisar el estado de la bandera C
; resultados > 0 (C=1)
MOV  R2,#0x4F 	;R2 = 0x4F   =79
MOV  R3,#0x39 	;R3 = 0x39   =57
SUBS R4,R2,R3 	;R4 = R2 – R3  = 22

MOV  R2, #0x8888
MOVT R2, #0x8888
MOV  R3, #0x3333
MOVT R3, #0x3333
SUBS R4,R2,R3 	;R4 = R2 – R3

; resultado < 0 (C=0)
MOV  R1,#0x4C 	;R1 = 0x4C   =76
MOV  R2,#0x6E 	;R2 = 0x6E   =110
SUBS R0,R1,R2 	;R0 = R1 – R2
; si es negativo N=1
; el resultado está en complemento a 2
; hay que negarlo y sumarle 1
; el resultante es el valor negativo

; Reverse Substract. util para obtener el complemento a 2 de un número
; invierte el orden de los operandos en una substraccion
; Sintaxis	RSB Rd,Rn,Op2 ;Rd = Op2 – Rn
MOV R1, #0x1 	;R1=1
RSB R0,R1,#0 	;R0= 0 – R1 = 0 – 1

MOV R1, #0x3 	;R1=0x03
RSB R0,R1,#0 	;R0= 0 – R1

;;;;;;;;;;;;;;;;;;;;;;;;;;;;;;;;;;;;;;;;;;;;;;;;;;;;;;;;;;;;;;;;;;;;;;;
;;; Operación Multiplicación

; multiplicacion usando MUL : Half_Word1 * Half_Word2 = 16bits*16bits=32bits
MOVW R0, #0xFFFF
MOVW R1, #2
MUL  R2, R0,R1

MOV R0, #0xFFFF
MOV R1, #0xFFFF
MUL R2, R0,R1

; multiplicacion de valores mayores a 16 bits: operandos deben estar en registros
MOVW R0, #0x86A0   ; R0= 100000
MOVT R0, #0x1
MOVW R1, #0x49F0   ; R1= 150000
MOVT R1, #0x2
MUL R3, R0, R1     ; solo se mantienen los 32 LSB R1*R0= 3 7E11 D600

; multiplicacion para resultados mayores a 32 bits (Unsigned MULtiplication)
UMULL R3,R4, R0,R1  ; RdHI,RdLO, Rn,Op2

; operacion MuLtiplicacion y Acumulación (esencial para procesamiento de señales), operandos en regs
; Sintaxis	MLA Rd,Rm,Rs,Rn ;Rd = (Rm × Rs) + Rn
MOV R1, #100 		;R1 = 100
MOV R2, #5 			;R2 = 5
MOV R3, #40 		;R3 = 40
MLA R4,R1,R2,R3 	;R4 = (R1 × R2) + R3 = (100 × 5) + 40 = 540

MOV R0, #4			; inicializa contador
MOV R1, #1
MOV R2, #2
MOV R4, #0
func1		        MLA R4,R1,R2,R4		; Multiplica y acumula:  R4 = R4 + (R1xR2)
ADDS R1,#1
ADDS R2,#2
SUBS R0,#1			; decrementa contador
BNE func1			; termina ciclo?

;;;;;;;;;;;;;;;;;;;;;;;;;;;;;;;;;;;;;;;;;;;;;;;;;;;;;;;;;;;;;;;;;;;;;;;
;;;        Instrucciones lógicas
MOVB R0, #10010000b
MOVB R1, #10000000b
ANDB R2,R0,R1		; AND

; EOR: Exclusive OR, enciende bits apagados o apaga bits encendidos = "Toggle", especificados por máscara
MOVB R3,#80			; R3=1000 0000
EOR R3,R3,#0x01    	; enciende bit 0, R3=0000 0001
EOR R3,R3,#0x01		; apaga bit 0,  R3=0000 0000
EOR R3,R3,#0x11		; enciede bit 4 y 0

; BIt Clear
BIC R3, #0x10		; apaga bit 4,  R3=0000 0001
EOR R3, #0x11		; enciende bit 4, apaga bit 1, R3=0001 0000

; ORR, permite encender bits de un registro sin modificar los demás
ORRS R3, #0x01


; preguntar si el bit 4 de R0 está encendido
; el valor del bit 4 se guarda en R2
ANDB R2,R0, #0x10	; R0 && mascara
ASRS R2,#4        	; desplaza R2 4 bits a la derecha, actualiza banderas
CMP R2,#1  		; el b4='1' ?

MOVB R0, #10010000b
TST  R0, #0x10	; prueba si todos los bits indicados en mascara son '0', afecta Z
TST  R0, #0x90	; la operacion es igual a ANDS pero no guarda resultado
TST  R0, #0x6F
TEQ  R0, #0x10	; prueba si el registro tiene encendidos los bits indicados por la máscara
TEQ  R0, #0x90	; la operacion es igual a EORS pero no guarda resultado

; Saltos condicionales a secciones de programas,
; Llamado a subrutinas y regreso de subrutinas
MOV  R0, #2
loop1		        SUBS R0,#1	; decrementa R0
CMP  R0,#0 	; es R0=0?
BEQ  func2  ; si, entonces "salta" a func2
NOP			; no, entonces sigue con programa
NOP
BL   func3	; "llama" a func3 y debe regresar
NOP			; sigue despues de ejecutar func3
NOP
B loop1		; repite loop1


func2       NOP			; ejecuta func2
NOP
B stop		; termina programa

func3		NOP			; ejecuta func2
NOP
BX LR       ; regresa a donde fué llamada




stop		B stop
.end

	\end{Verbatim}
	
	
	
\end{enumerate}


 \section{Desarrollo.}

 Escriba un programa que funcione como una calculadora de números enteros de hasta 32 bits, con operaciones de Suma, Resta, Multiplicación, AND, OR, XOR. 

 Para elegir la operación a realizar, desde el depurador se escribirá un valor en R0 del 1 al 6 (1:suma, 2:resta, etc.). El programa debe hacer llamados a dos subrutinas: una para carga de operandos y otra para guardar resultado.
 
  Los dos operandos se encuentran definidos en memoria RAM y se deben colocar en R1 y R2. El resultado se escribe también en memoria RAM, usando R3. 
  
  Por tanto se debe hacer acceso de lectura y escritura (LDR, STR). 
  
  Los operandos se pueden modificar en cualquier sobre-escribiendo su valor en memoria RAM. 


  
  \section{Cuestionario}
  
  \begin{enumerate}
  	\item Después de ejecutar la instrucción SUBS, ¿qué signo tiene el resultado cuando las bandera C = 0, y cuando C=1? En cada caso, qué valor debe tener la bandera N? 
  	
  	Para un valor en un Registro, en lenguaje ensamblador.

  	\item ¿Cómo se enciende un bit específico sin afectar a los demás? (2 maneras).

  	\item ¿Cómo se apaga un bit específico sin afectar a los demás? (2 maneras).
  	
  	\item ¿Cómo se invierte el estado de un bit específico?.
  	
  	\item Al emplear instrucción de comparación CMP, ¿Qué nos dice la bandera Z?.
  	
  	\item Se quiere saber si los 4 LSB de un byte son ‘0’, ¿cómo emplearía la instrucción TST y qué condición/bandera se debe cumplir/establecer?.
  	
  	\item Se quiere saber si los 4 MSB de un byte son ‘1’, ¿cómo emplearía la instrucción TEQ y qué condición/bandera se debe cumplir/establecer? 
  	
  	\item Para qué sirve la directiva .retain 
  	  
  \end{enumerate}

\section{Conclusiones.}
  \begin{thebibliography}{}                           % Bibliografia
    \bibitem{ref:cita}                                % Etiqueta con la que se hara la referencia o cita.
      Como citar: \url{http://www.cva.itesm.mx/biblioteca/pagina_con_formato_version_oct/apa.htm} % URL de apoyo para citar.

    \bibitem{ref:web1}
      Autor,
      (Fecha de publicacion),
      Titulo, paginas,
      Fecha de recuperacion,
      Sitio web: \url{http://www.google.com}

    \bibitem{ref:github}
      Repositorio del proyecto \url{https://github.com/penserbjorne}
  \end{thebibliography}

%%%%%%%%%%%%%%%%%%%%%%%%%%%%%%%%%%%%%%%%%%%%%%%%%%%%%%%%%%%%%%%%%%%%%%%%%%%%%%%%%%%%%%%%%

\end{document}
