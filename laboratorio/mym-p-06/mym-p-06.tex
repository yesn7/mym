%%%%%%%%%%%%%%%%%%%%%%%%%%%%%%%%%%%%%%%%%%%%%%%%%%%%%%%%%%%%%%%%%%%%%%%%%%%%%%%%%%%%%%%%%
% Autor:        Aguilar Enriquez, Paul Sebastian a.k.a. Penserbjorne
% Fecha:        05/02/2017
% Descripcion:  Plantilla base para actividades o tareas.
%%%%%%%%%%%%%%%%%%%%%%%%%%%%%%%%%%%%%%%%%%%%%%%%%%%%%%%%%%%%%%%%%%%%%%%%%%%%%%%%%%%%%%%%%

\documentclass[a4paper,11pt]{article}                 % Papel tamaño carta, texto de 11pt.

\usepackage[top=2cm, bottom=2cm, left=2.2cm, right=2.2cm]{geometry} % Margenes
\usepackage[T1]{fontenc}                              % Indicamos la codificacion de las fuentes.
\usepackage[utf8x]{inputenc}                          % Definimos la codificacion.
\usepackage{lmodern}                                  % Para poder usar acentos.
\usepackage[spanish]{babel}                           % Usaremos idioma español.
\usepackage{amsmath}                                  % Para formulas matematicas.
\usepackage{graphicx}                                 % Para imagenes.
\usepackage{float}                                    % Para posicionar objetos.
\usepackage{booktabs}                                 % Para formatear tablas.
\usepackage{hyperref}                                 % Para enlaces y referencias.
\usepackage{enumerate} 
%\usepackage{colortbl}
%%%%%%%%%%%%%%%%%%%%%%%%%%%%%%%%%%%%%%%%%%%%%%%%%%%%%%%%%%%%%%%%%%%%%%%%%%%%%%%%%%%%%%%%%

% Los logos tienen posiciones relativas al nombre de la escuela.
% Cada imagen esta desplazada con respecto al texto, en este caso nombre de la univseridad.
% No se necesitan paquetes adicionales, el entorno estandar para imagenes de LaTeX puede hacerlo.
% El truco esta en definir una imagen de tamaño cero, asi no afecta al centrar los titulos.
\def\logoUNAM{%
  \begin{picture}(0,0)\unitlength=1cm
    \put (-3.5,-3) {\includegraphics[width=8em]{images/escudo-unam}}
  \end{picture}
}

\def\logoFI{%
  \begin{picture}(0,0)\unitlength=1cm
    \put (0.5,-3) {\includegraphics[width=8em]{images/escudo-fi}}
  \end{picture}
}

%%%%%%%%%%%%%%%%%%%%%%%%%%%%%%%%%%%%%%%%%%%%%%%%%%%%%%%%%%%%%%%%%%%%%%%%%%%%%%%%%%%%%%%%%

\author{Pérez Navarro Maria Yesica - 414039694}  % Autor de la actividad.
\title{Previo 06: Programación en lenguaje\\ Assembly (Ensamblador) 3}                % Titulo de la actividad.
\date{dd/mm/yyyy}                                           % Fecha de entrega.
\def\universidad{Universidad Nacional Autónoma de México}   % Nombre de la universidad.
\def\facultad{Facultad de Ingeniería}                              % Nombre de la facultdad.
\def\semestre{2018-1}                                     % Semestre lectivo.
\def\materia{Lab. Microcontroladores y Microprocesadores - Grupo 03}               % Nombre de la materia y grupo.
\makeatletter

%%%%%%%%%%%%%%%%%%%%%%%%%%%%%%%%%%%%%%%%%%%%%%%%%%%%%%%%%%%%%%%%%%%%%%%%%%%%%%%%%%%%%%%%%

\begin{document}
  
  % Titulo del documento con logos.
  \begin{center}
    \logoUNAM {\Large \universidad} \logoFI\par
    {\large \facultad}\par
    \semestre\par
    \materia\par
    \@author\par
    \@date\par
    \@title
  \end{center}

  \hrulefill\par

  \pagenumbering{gobble}                              % Oculta el numero de pagina.
%  \tableofcontents                                    % Crea el indice o tabla de contenido.

%%%%%%%%%%%%%%%%%%%%%%%%%%%%%%%%%%%%%%%%%%%%%%%%%%%%%%%%%%%%%%%%%%%%%%%%%%%%%%%%%%%%%%%%%

  %\newpage                                            % Inserta una pagina nueva.
  \pagenumbering{arabic}                              % Muestra el numero de pagina.
  
  \section{Seguridad en la Ejecución.}
  \begin{table}[H]
  	\begin{tabular}{|l|l|l|}
  		\hline
  		 & Peligro o fuente de energía & Riesgo asociado  \\ \hline
  		1 & Manejo de Corriente Alterna &Electrochoque    \\ \hline
  		2 & Manejo de corriente Continua & Daño al equipo \\ \hline
  	\end{tabular}
  	\centering
  \end{table}

\section{Objetivos de aprendizaje.}
\begin{itemize}
	\item El alumno desarrollará habilidades para la manipulación de datos: bytes, nibbles y bits, haciendo uso de operaciones lógicas como sumas, restas, AND y OR, desplazamientos lógicos, así como reforzar el uso de saltos a subrutinas y llamado a subrutinas y regreso al programa principal. 
\end{itemize}

\section{Material y equipo.}
 
\begin{itemize}
	\item Tarjeta de desarrollo.
	\item CCS IDE.
	\item Tabla de caracteres ASCII. 
\end{itemize}
  

  
\section{Actividad previa.}                   

\begin{enumerate}[a)]

\item ¿En qué consiste la representación BCD? 

\item  Investigue cómo se guarda en memoria una cadena de texto en lenguaje C (incluyendo formato e indicador  de fin de cadena). 

\item  Describa cómo se convierte una cifra numérica a cadena ASCII y viceversa (qué operación se realiza)

\item  Convierta el número 0x80 (128 decimal), a un conjunto de 3 caracteres ASCII. 
	
\end{enumerate}


 \section{Desarrollo.}

Escriba un programa que tenga 3 funciones: 

\begin{itemize}
	\item  Convertir un valor BCD de 4 cifras en su representación ASCII. Por ejemplo, el número 2017 (0x07E1 en hexadecimal), en cuatro cifras BCD: 0x2,0x0,0x1,0x7, en cuatro dígitos ASCII: 0x32,0x30,0x31,0x37. Se forzará el almacenamiento de cada carácter en espacio de un byte, evitando escribir en un espacio de 32 bits y ahorrar espacio de memoria.) 
	\item  Convertir una cadena de texto de 4 cifras (caracteres ASCII) en su representación numérica. Por ejemplo la cadena “2017” (0x32,0x30,0x31,0x37), convertirla a un número de 16 bits), empleando la ponderación de la posición de cada dígito en el sistema de numeración decimal, por ejemplo:   
	
	\item  Empaquetar en su valor real, el valor BCD de 4 dígitos (2017 0x07E1), y contar el número de ‘1’ que tiene el valor numérico en su representación binaria.  
\end{itemize}

Requerimientos:

Se definen en RAM las variables a emplear:
\begin{itemize}
	\item R0 se usará como el contenedor para la opción del menú: 
	
	\item El dato numérico BCD de entrada se escribe en una localidad de RAM. El dato en ASCII de salida se escribe en RAM y se agrega el fin de cadena. 
	\item El dato en ASCII de entrada se escribe en máximo 5 Bytes en RAM ya incluido el fin de cadena. El dato numérico BCD de salida se escribe en una localidad de RAM.
	\item El valor numérico resultante de empaquetar el BCD se escribe en RAM -El resultado del número de bits en ‘1’ se escribe en RAM. 
\end{itemize}
De forma similar a la práctica anterior, se necesita hacer lectura y escritura en RAM. Puede usar funciones y llamado a funciones o saltos a rutinas.  


 


  
  \section{Cuestionario}
  
  \begin{enumerate}
  	 
  	
  	\item ¿Cuál es la diferencia entre códigos BCD y ASCII?

  	\item ¿Cómo convierte un número a representación BCD?

  	\item ¿Qué operaciones se realizan para convertir cifras BCD a ASCII y viceversa? 
  	
  	\item Escriba un algoritmo para representar un valor numérico en BCD, y éste a ASCII; por ejemplo el valor 109 (0x6D en hexadecimal), representado en 3 bytes como 0x01, 0x00, 0x09. 
  	  
  \end{enumerate}

\section{Conclusiones.}
  \begin{thebibliography}{}                           % Bibliografia
    \bibitem{ref:cita}                                % Etiqueta con la que se hara la referencia o cita.
      Como citar: \url{http://www.cva.itesm.mx/biblioteca/pagina_con_formato_version_oct/apa.htm} % URL de apoyo para citar.

    \bibitem{ref:web1}
      Autor,
      (Fecha de publicacion),
      Titulo, paginas,
      Fecha de recuperacion,
      Sitio web: \url{http://www.google.com}

    \bibitem{ref:github}
      Repositorio del proyecto \url{https://github.com/penserbjorne}
  \end{thebibliography}

%%%%%%%%%%%%%%%%%%%%%%%%%%%%%%%%%%%%%%%%%%%%%%%%%%%%%%%%%%%%%%%%%%%%%%%%%%%%%%%%%%%%%%%%%

\end{document}
